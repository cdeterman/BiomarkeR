% \VignetteIndexEntry{ A Short Introduction to the OmicMarkeR Package}
% \VignettePackage{OmicsMarkeR}
%\VignetteEngine{knitr::knitr}

% To compile this document
% library('knitr'); rm(list=ls()); knit('OmicsMarkeR.Rnw')

\documentclass[12pt]{article}\usepackage[]{graphicx}\usepackage[usenames,dvipsnames]{color}
%% maxwidth is the original width if it is less than linewidth
%% otherwise use linewidth (to make sure the graphics do not exceed the margin)
\makeatletter
\def\maxwidth{ %
  \ifdim\Gin@nat@width>\linewidth
    \linewidth
  \else
    \Gin@nat@width
  \fi
}
\makeatother

\definecolor{fgcolor}{rgb}{0.345, 0.345, 0.345}
\newcommand{\hlnum}[1]{\textcolor[rgb]{0.686,0.059,0.569}{#1}}%
\newcommand{\hlstr}[1]{\textcolor[rgb]{0.192,0.494,0.8}{#1}}%
\newcommand{\hlcom}[1]{\textcolor[rgb]{0.678,0.584,0.686}{\textit{#1}}}%
\newcommand{\hlopt}[1]{\textcolor[rgb]{0,0,0}{#1}}%
\newcommand{\hlstd}[1]{\textcolor[rgb]{0.345,0.345,0.345}{#1}}%
\newcommand{\hlkwa}[1]{\textcolor[rgb]{0.161,0.373,0.58}{\textbf{#1}}}%
\newcommand{\hlkwb}[1]{\textcolor[rgb]{0.69,0.353,0.396}{#1}}%
\newcommand{\hlkwc}[1]{\textcolor[rgb]{0.333,0.667,0.333}{#1}}%
\newcommand{\hlkwd}[1]{\textcolor[rgb]{0.737,0.353,0.396}{\textbf{#1}}}%

\usepackage{framed}
\makeatletter
\newenvironment{kframe}{%
 \def\at@end@of@kframe{}%
 \ifinner\ifhmode%
  \def\at@end@of@kframe{\end{minipage}}%
  \begin{minipage}{\columnwidth}%
 \fi\fi%
 \def\FrameCommand##1{\hskip\@totalleftmargin \hskip-\fboxsep
 \colorbox{shadecolor}{##1}\hskip-\fboxsep
     % There is no \\@totalrightmargin, so:
     \hskip-\linewidth \hskip-\@totalleftmargin \hskip\columnwidth}%
 \MakeFramed {\advance\hsize-\width
   \@totalleftmargin\z@ \linewidth\hsize
   \@setminipage}}%
 {\par\unskip\endMakeFramed%
 \at@end@of@kframe}
\makeatother

\definecolor{shadecolor}{rgb}{.97, .97, .97}
\definecolor{messagecolor}{rgb}{0, 0, 0}
\definecolor{warningcolor}{rgb}{1, 0, 1}
\definecolor{errorcolor}{rgb}{1, 0, 0}
\newenvironment{knitrout}{}{} % an empty environment to be redefined in TeX

\usepackage{alltt}


\RequirePackage{C:/Users/cdeterman/Documents/R/R-3.2.0/library/BiocStyle/resources/latex/Bioconductor}

\AtBeginDocument{\bibliographystyle{C:/Users/cdeterman/Documents/R/R-3.2.0/library/BiocStyle/resources/latex/unsrturl}}



\bioctitle{A Short Introduction to the \Biocpkg{OmicsMarkeR} Package}
\author{Charles Determan Jr.\footnote{cdetermanjr@gmail.com}}
\IfFileExists{upquote.sty}{\usepackage{upquote}}{}
\begin{document}

\maketitle
\thispagestyle{empty}

\maketitle
\section{Introduction}
The \Biocpkg{OmicsMarkeR} package contains functions to streamline the analysis
of 'omics' level datasets with the objective to classify groups and determine 
the most important features. \Biocpkg{OmicsMarkeR} loads packages as needed and 
assumes that they are installed.  I will provide a short tutorial using the 
both synthetic datasets created by internal functions as well as the 'Sonar' 
dataset. 

Install \Biocpkg{OmicsMarkeR} using  
\begin{knitrout}
\definecolor{shadecolor}{rgb}{0.969, 0.969, 0.969}\color{fgcolor}\begin{kframe}
\begin{alltt}
\hlkwd{source}\hlstd{(}\hlstr{"http://bioconductor.org/biocLite.R"}\hlstd{)}
\hlkwd{biocLite}\hlstd{(}\hlstr{"OmicsMarkeR"}\hlstd{)}
\end{alltt}
\end{kframe}
\end{knitrout}
    to ensure that all the needed packages are installed.

\newpage
\maketitle
\section{Basic Classification Example}
\Biocpkg{OmicsMarkeR} has a few simplified functions that attempt to streamline 
the classification and feature selection process including the addition of 
stability metrics. We will first generate a synthetic dataset.  This includes 
three functions that can be used to create multivariate datasets that can 
mimic specific omics examples.  This can include a null dataset via the
\Rfunction{create.rand.matrix} to create a random multivarate dataset with
\Rcode{nvar = 50} and \Rcode{nsamp = 100}.  The \Rfunction{create.corr.matrix} 
function induces correlations to the datasets.  The 
\Rfunction{create.discr.matrix} function induces variables to be discriminate 
between groups.  The number of groups can be specified with \Rcode{num.groups}. 

\begin{knitrout}
\definecolor{shadecolor}{rgb}{0.969, 0.969, 0.969}\color{fgcolor}\begin{kframe}
\begin{alltt}
\hlkwd{library}\hlstd{(}\hlstr{"OmicsMarkeR"}\hlstd{)}
\hlkwd{set.seed}\hlstd{(}\hlnum{123}\hlstd{)}
\hlstd{dat.discr} \hlkwb{<-} \hlkwd{create.discr.matrix}\hlstd{(}
    \hlkwd{create.corr.matrix}\hlstd{(}
        \hlkwd{create.random.matrix}\hlstd{(}\hlkwc{nvar} \hlstd{=} \hlnum{50}\hlstd{,}
                             \hlkwc{nsamp} \hlstd{=} \hlnum{100}\hlstd{,}
                             \hlkwc{st.dev} \hlstd{=} \hlnum{1}\hlstd{,}
                             \hlkwc{perturb} \hlstd{=} \hlnum{0.2}\hlstd{)),}
    \hlkwc{D} \hlstd{=} \hlnum{10}
\hlstd{)}
\end{alltt}
\end{kframe}
\end{knitrout}
    
To avoid confusion in the coding, one can isolate the variables and classes 
from the newly created synthetic dataset.  These two objects are then used in 
the \Rfunction{fs.stability} function.  I can then choose which algorithm(s) to 
apply e.g. \Rcode{method = c("plsda", "rf")}, the number of top important 
features \Rcode{f = 20}, the number of bootstrap reptititions for stability 
analysis \Rcode{k = 3}, the number of k-fold cross-validations 
\Rcode{k.folds = 10} and if I would like to see the progress output 
\Rcode{verbose = TRUE}.

\warning{You will receive warnings if you run code exactly as shown
in this vignette.  This is intentional as the PLSDA runs with this simple
dataset often only need a single component but you need to indicate 2
to fit the model.  This is provided for the users information.}

\begin{knitrout}
\definecolor{shadecolor}{rgb}{0.969, 0.969, 0.969}\color{fgcolor}\begin{kframe}
\begin{alltt}
\hlstd{vars} \hlkwb{<-} \hlstd{dat.discr}\hlopt{$}\hlstd{discr.mat}
\hlstd{groups} \hlkwb{<-} \hlstd{dat.discr}\hlopt{$}\hlstd{classes}
\hlstd{fits} \hlkwb{<-} \hlkwd{fs.stability}\hlstd{(vars,}
                    \hlstd{groups,}
                    \hlkwc{method} \hlstd{=} \hlkwd{c}\hlstd{(}\hlstr{"plsda"}\hlstd{,} \hlstr{"rf"}\hlstd{),}
                    \hlkwc{f} \hlstd{=} \hlnum{10}\hlstd{,}
                    \hlkwc{k} \hlstd{=} \hlnum{3}\hlstd{,}
                    \hlkwc{k.folds} \hlstd{=} \hlnum{10}\hlstd{,}
                    \hlkwc{verbose} \hlstd{=} \hlstr{'none'}\hlstd{)}
\end{alltt}


{\ttfamily\noindent\color{warningcolor}{\#\# Warning: package 'plyr' was built under R version 3.2.1}}

{\ttfamily\noindent\itshape\color{messagecolor}{\#\# randomForest 4.6-10\\\#\# Type rfNews() to see new features/changes/bug fixes.}}

{\ttfamily\noindent\color{warningcolor}{\#\# Warning: package 'e1071' was built under R version 3.2.1}}

{\ttfamily\noindent\itshape\color{messagecolor}{\#\# Loading required package: survival}}

{\ttfamily\noindent\color{warningcolor}{\#\# Warning: package 'survival' was built under R version 3.2.1}}

{\ttfamily\noindent\itshape\color{messagecolor}{\#\# Loading required package: lattice}}

{\ttfamily\noindent\color{warningcolor}{\#\# Warning: package 'lattice' was built under R version 3.2.1}}

{\ttfamily\noindent\itshape\color{messagecolor}{\#\# Loading required package: splines\\\#\# Loading required package: parallel\\\#\# Loaded gbm 2.1.1\\\#\# Loading required package: cluster}}

{\ttfamily\noindent\color{warningcolor}{\#\# Warning: package 'cluster' was built under R version 3.2.1}}

{\ttfamily\noindent\itshape\color{messagecolor}{\#\# Loading required package: Matrix}}

{\ttfamily\noindent\color{warningcolor}{\#\# Warning: package 'Matrix' was built under R version 3.2.1}}

{\ttfamily\noindent\itshape\color{messagecolor}{\#\# Loaded glmnet 2.0-2}}

{\ttfamily\noindent\color{warningcolor}{\#\# Warning in training(data = trainData, method = method[i], tuneValue = as.data.frame(bestTune[[i]]), : PLSDA model contained only 1 component. \\\#\#\ \ \ \ \ \ \ \ \ \ \ \ \ \ \ \  PLSDA requires at least 2 components.\\\#\#\ \ \ \ \ \ \ \ \ \ \ \ \ \ \ \  \\\#\# Model fit with 2 components}}

{\ttfamily\noindent\color{warningcolor}{\#\# Warning in training(data = trainData, method = method[i], tuneValue = as.data.frame(bestTune[[i]]), : PLSDA model contained only 1 component. \\\#\#\ \ \ \ \ \ \ \ \ \ \ \ \ \ \ \  PLSDA requires at least 2 components.\\\#\#\ \ \ \ \ \ \ \ \ \ \ \ \ \ \ \  \\\#\# Model fit with 2 components}}

{\ttfamily\noindent\color{warningcolor}{\#\# Warning in training(data = trainData, method = method[d], tuneValue = tuned.methods\$bestTune[[d]], : PLSDA model contained only 1 component. \\\#\#\ \ \ \ \ \ \ \ \ \ \ \ \ \ \ \  PLSDA requires at least 2 components.\\\#\#\ \ \ \ \ \ \ \ \ \ \ \ \ \ \ \  \\\#\# Model fit with 2 components}}

{\ttfamily\noindent\color{warningcolor}{\#\# Warning in training(data = trainData.new[[d]], method = method[d], tuneValue = as.data.frame(t(unlist(tunedModel.new[[d]]\$bestTune))), : PLSDA model contained only 1 component. \\\#\#\ \ \ \ \ \ \ \ \ \ \ \ \ \ \ \  PLSDA requires at least 2 components.\\\#\#\ \ \ \ \ \ \ \ \ \ \ \ \ \ \ \  \\\#\# Model fit with 2 components}}

{\ttfamily\noindent\color{warningcolor}{\#\# Warning in training(data = trainData, method = method[d], tuneValue = tuned.methods\$bestTune[[d]], : PLSDA model contained only 1 component. \\\#\#\ \ \ \ \ \ \ \ \ \ \ \ \ \ \ \  PLSDA requires at least 2 components.\\\#\#\ \ \ \ \ \ \ \ \ \ \ \ \ \ \ \  \\\#\# Model fit with 2 components}}

{\ttfamily\noindent\color{warningcolor}{\#\# Warning in training(data = trainData.new[[d]], method = method[d], tuneValue = as.data.frame(t(unlist(tunedModel.new[[d]]\$bestTune))), : PLSDA model contained only 1 component. \\\#\#\ \ \ \ \ \ \ \ \ \ \ \ \ \ \ \  PLSDA requires at least 2 components.\\\#\#\ \ \ \ \ \ \ \ \ \ \ \ \ \ \ \  \\\#\# Model fit with 2 components}}\end{kframe}
\end{knitrout}
    
If I would like to see the performance metrics, I can simply use the 
\Rfunction{performance.metrics} function.  This will provide a concise 
data.frame of confusion matrix and ROC statistics.  Additionally, the 
Robustness-Performance Trade-off value (RPT) is provided with the results.

\begin{knitrout}
\definecolor{shadecolor}{rgb}{0.969, 0.969, 0.969}\color{fgcolor}\begin{kframe}
\begin{alltt}
\hlkwd{performance.metrics}\hlstd{(fits)}
\end{alltt}
\begin{verbatim}
##                        plsda         rf
## Accuracy          0.88333333 0.93333333
## Kappa             0.76666667 0.86666667
## ROC.AUC           0.92000000 0.96000000
## Sensitivity       0.86666667 0.93333333
## Specificity       0.90000000 0.93333333
## Pos Pred Value    0.90303030 0.93333333
## Neg Pred Value    0.87777778 0.93333333
## Accuracy SD       0.07637626 0.11547005
## Kappa SD          0.15275252 0.23094011
## ROC.AUC SD        0.07549834 0.06928203
## Sensitivity SD    0.11547005 0.11547005
## Specificity SD    0.10000000 0.11547005
## Pos Pred Value SD 0.10013765 0.11547005
## Neg Pred Value SD 0.10715168 0.11547005
\end{verbatim}
\begin{alltt}
\hlstd{fits}\hlopt{$}\hlstd{RPT}
\end{alltt}
\begin{verbatim}
##     plsda        rf 
## 0.7933472 0.3017964
\end{verbatim}
\end{kframe}
\end{knitrout}

If I would want to see the occurance of the features the model identified as the
most important, this is accomplished by \Rfunction{feature.table}.  This 
function returns a simple table reporting the consistency (i.e. how many times
identified) and frequency (percent identified in all runs).

\begin{knitrout}
\definecolor{shadecolor}{rgb}{0.969, 0.969, 0.969}\color{fgcolor}\begin{kframe}
\begin{alltt}
\hlkwd{feature.table}\hlstd{(fits,} \hlstr{"plsda"}\hlstd{)}
\end{alltt}
\begin{verbatim}
##    features consistency frequency
## 1        V1           3         1
## 2       V11           3         1
## 3       V14           3         1
## 4       V21           3         1
## 5       V23           3         1
## 6       V26           3         1
## 7       V28           3         1
## 8       V37           3         1
## 9       V16           2     0.667
## 10       V2           1     0.333
## 11      V47           1     0.333
## 12      V41           1     0.333
## 13      V44           1     0.333
\end{verbatim}
\end{kframe}
\end{knitrout}


If the user is interesting in applying the fitted model (determined by 
\Rfunction{fs.stability}) towards some new data this can be accomplished with
\Rfunction{predictNewClasses}.  This could either be yet another level to 
evaluate the tuned model's performance or if the user is applying the model
in a production type setting where you are systematically using this model
on new data that comes in.

\begin{knitrout}
\definecolor{shadecolor}{rgb}{0.969, 0.969, 0.969}\color{fgcolor}\begin{kframe}
\begin{alltt}
\hlcom{# create some 'new' data}
\hlstd{newdata} \hlkwb{<-} \hlkwd{create.discr.matrix}\hlstd{(}
    \hlkwd{create.corr.matrix}\hlstd{(}
        \hlkwd{create.random.matrix}\hlstd{(}\hlkwc{nvar} \hlstd{=} \hlnum{50}\hlstd{,}
                             \hlkwc{nsamp} \hlstd{=} \hlnum{100}\hlstd{,}
                             \hlkwc{st.dev} \hlstd{=} \hlnum{1}\hlstd{,}
                             \hlkwc{perturb} \hlstd{=} \hlnum{0.2}\hlstd{)),}
    \hlkwc{D} \hlstd{=} \hlnum{10}
\hlstd{)}\hlopt{$}\hlstd{discr.mat}

\hlcom{# original data combined to a data.frame}
\hlstd{orig.df} \hlkwb{<-} \hlkwd{data.frame}\hlstd{(vars, groups)}

\hlcom{# see what the PLSDA predicts for the new data}
\hlcom{# NOTE, newdata does not require a .classes column}
\hlkwd{predictNewClasses}\hlstd{(fits,} \hlstr{"plsda"}\hlstd{, orig.df, newdata)}
\end{alltt}
\end{kframe}
\end{knitrout}

\textbf{Note} - This function is not as efficient as I would like at present.
Currently it requires the original dataset to refit the model from the 
parameters retained from \Rfunction{fs.stability}.  I intend to provide the
option to retain fitted models so a user can simply pull them without a need
to refit.  However, I am concerned about the potential size of objects (e.g. 
random forest, gbm, etc.).  Thoughts and contributions are welcome.


\newpage
\maketitle
\section{Ensemble Methods}

In machine learning ensembles of models can be used to obtain better predictive
performance than any individual model.  There are multiple types of ensemble
types including the bayes optimal classifier, bayesian model averaging (BMA),
bayesian model combination (BMC), boostrap aggregation (bagging), and boosting. 
Although it is the intention to include all the methods the only currently
implemented method is bagging.

Bagging, in the simplest terms, is defined as giving a set of trained models
equal weight when 'voting' on an optimal solution.  The most familiar 
application of this concept is in the random forest algorithm.  This technique
requires a defined aggregation technique to combine the set of models.  
Implemented methods include Complete Linear Aggregation (\Rfunction{CLA}), 
Ensemble Mean (\Rfunction{EM}), Ensemble Stability (\Rfunction{ES}), and
Ensemble Exponential (\Rfunction{EE}).

To conduct an ensemble analysis, the code is nearly identical to the first
example in Section 1.  The \Rfunction{fs.ensembl.stability} function contains
the same arguments as \Rfunction{fs.stability} in addition to a few more for
the ensemble components.  Please see \Rcode{?fs.ensemble.stability} for
complete details on each.  The two major additional parameters are 
\Robject{bags} and \Robject{aggregation.metric}.  The \Robject{bags} parameter
naturally defines the number of bagging iterations and the 
\Robject{aggregation.metric} is a string defining the aggregation method.  These
arguments have common defaults predefined so the call can be:

\begin{knitrout}
\definecolor{shadecolor}{rgb}{0.969, 0.969, 0.969}\color{fgcolor}\begin{kframe}
\begin{alltt}
\hlstd{fits} \hlkwb{<-} \hlkwd{fs.ensembl.stability}\hlstd{(vars,}
                            \hlstd{groups,}
                            \hlkwc{method} \hlstd{=} \hlkwd{c}\hlstd{(}\hlstr{"plsda"}\hlstd{,} \hlstr{"rf"}\hlstd{),}
                            \hlkwc{f} \hlstd{=} \hlnum{10}\hlstd{,}
                            \hlkwc{k} \hlstd{=} \hlnum{3}\hlstd{,}
                            \hlkwc{k.folds} \hlstd{=} \hlnum{10}\hlstd{,}
                            \hlkwc{verbose} \hlstd{=} \hlstr{'none'}\hlstd{)}
\end{alltt}
\end{kframe}
\end{knitrout}

As in Section 1, the \Rfunction{performance.metrics} function can be applied
for summary statistics.

\newpage
\subsection{Aggregation Methods}

If the user wishes to apply an aggregation method manually utilizing results
from an alternative analysis, this can also be done.  This package provides
a the wrapper \Rfunction{aggregation} to apply this analysis.  For these 
methods, the variables must have been ranked and the rownames assigned as the
variable names.  The function then will return that aggregated list of 
variables with their new respective ranks.

\begin{knitrout}
\definecolor{shadecolor}{rgb}{0.969, 0.969, 0.969}\color{fgcolor}\begin{kframe}
\begin{alltt}
\hlcom{# test data}
\hlstd{ranks} \hlkwb{<-} \hlkwd{replicate}\hlstd{(}\hlnum{5}\hlstd{,} \hlkwd{sample}\hlstd{(}\hlkwd{seq}\hlstd{(}\hlnum{50}\hlstd{),} \hlnum{50}\hlstd{))}
\hlkwd{row.names}\hlstd{(ranks)} \hlkwb{<-} \hlkwd{paste0}\hlstd{(}\hlstr{"V"}\hlstd{,} \hlkwd{seq}\hlstd{(}\hlnum{50}\hlstd{))}

\hlkwd{head}\hlstd{(}\hlkwd{aggregation}\hlstd{(ranks,} \hlstr{"CLA"}\hlstd{))}
\end{alltt}
\begin{verbatim}
##     [,1]
## V45  1.0
## V7   2.0
## V50  3.0
## V18  4.0
## V15  5.0
## V20  6.5
\end{verbatim}
\end{kframe}
\end{knitrout}

This is used internally in \Rfunction{fs.ensembl.stability} to optimize which
variables are selected to be included in the final optimized model.  The only
exception to the format above is with the \Rfunction{EE} function where the 
number of variables must be defined with \Robject{f}.

\newpage
\maketitle
\section{Custom Tuning}

The default implementation assumes that each model will be tuned with the same
resolution of tuning parameters.  For example, a default call with PLSDA and 
Random Forest will result in tuning 3 components and 3 mtry for each
respectively.  However, let's say I want to be more fine tuned with my
random forest model.  You can create a customized grid using 
\Rfunction{denovo.grid}.

\begin{knitrout}
\definecolor{shadecolor}{rgb}{0.969, 0.969, 0.969}\color{fgcolor}\begin{kframe}
\begin{alltt}
\hlcom{# requires data.frame of variables and classes}
\hlstd{plsda} \hlkwb{<-} \hlkwd{denovo.grid}\hlstd{(orig.df,} \hlstr{"plsda"}\hlstd{,} \hlnum{3}\hlstd{)}
\hlstd{rf} \hlkwb{<-} \hlkwd{denovo.grid}\hlstd{(orig.df,} \hlstr{"rf"}\hlstd{,} \hlnum{5}\hlstd{)}

\hlcom{# create grid list}
\hlcom{# Make sure to assign appropriate model names}
\hlstd{grid} \hlkwb{<-} \hlkwd{list}\hlstd{(}\hlkwc{plsda}\hlstd{=plsda,} \hlkwc{rf}\hlstd{=rf)}

\hlcom{# pass to fs.stability or fs.ensemble.stability}
\hlstd{fits} \hlkwb{<-} \hlkwd{fs.stability}\hlstd{(vars,}
                    \hlstd{groups,}
                    \hlkwc{method} \hlstd{=} \hlkwd{c}\hlstd{(}\hlstr{"plsda"}\hlstd{,} \hlstr{"rf"}\hlstd{),}
                    \hlkwc{f} \hlstd{=} \hlnum{10}\hlstd{,}
                    \hlkwc{k} \hlstd{=} \hlnum{3}\hlstd{,}
                    \hlkwc{k.folds} \hlstd{=} \hlnum{10}\hlstd{,}
                    \hlkwc{verbose} \hlstd{=} \hlstr{'none'}\hlstd{,}
                    \hlkwc{grid} \hlstd{= grid)}
\end{alltt}
\end{kframe}
\end{knitrout}

The user can create their own grid completely manually but must use the 
appropriate names as defined by the functions.  These can be check with 
\Rfunction{params}.  As an example: \Rcode{params{method="plsda"}}.  

\textbf{Note} - the argument names must be preceded by a period.  This is to
prevent any unforseen conflicts in the code.

\newpage
\maketitle
\section{Stability Metrics}
\subsection{Stability Metric Basics}

It is quite possible that a user may already have fitted a model previously
or using a model that has not yet been implemented in this package.  However,
they may be interested in applying one or more of the stability metrics
defined within this package.  These functions are very simple to use.  To 
demonstrate, let's create some sample data consisting of our \textbf{Metabolite
Population}.

\begin{knitrout}
\definecolor{shadecolor}{rgb}{0.969, 0.969, 0.969}\color{fgcolor}\begin{kframe}
\begin{alltt}
\hlstd{metabs} \hlkwb{<-} \hlkwd{paste}\hlstd{(}\hlstr{"Metabolite"}\hlstd{,} \hlkwd{seq}\hlstd{(}\hlnum{20}\hlstd{),} \hlkwc{sep}\hlstd{=}\hlstr{"_"}\hlstd{)}
\end{alltt}
\end{kframe}
\end{knitrout}

Now, let's say you have run you different special model twice on different 
samples of a dataset.  You complete your feature selection and get two lists
of Metabolites.  Here I am just randomly sampling.

\begin{knitrout}
\definecolor{shadecolor}{rgb}{0.969, 0.969, 0.969}\color{fgcolor}\begin{kframe}
\begin{alltt}
\hlkwd{set.seed}\hlstd{(}\hlnum{13}\hlstd{)}
\hlstd{run1} \hlkwb{<-} \hlkwd{sample}\hlstd{(metabs,} \hlnum{10}\hlstd{)}
\hlstd{run2} \hlkwb{<-} \hlkwd{sample}\hlstd{(metabs,} \hlnum{10}\hlstd{)}
\end{alltt}
\end{kframe}
\end{knitrout}

The user can now evaluate how similar the sets of metabolites selected are via
multiple possible stability metrics.  These include the Jaccard Index 
(\Rfunction{jaccard}), Dice-Sorensen (\Rfunction{sorensen}), Ochiai's Index 
(\Rfunction{ochiai}), Percent of Overlapping Features (\Rfunction{pof}), 
Kuncheva's Index (\Rfunction{kuncheva}), Spearman (\Rfunction{spearman}), and
Canberra (\Rfunction{canberra}).  The latter two methods are not 
\emph{Set Methods} and do require the same number of features in each vector.
The relevant citations are provided in each function's documenation.

The general use for most of the functions is:

\begin{knitrout}
\definecolor{shadecolor}{rgb}{0.969, 0.969, 0.969}\color{fgcolor}\begin{kframe}
\begin{alltt}
\hlkwd{jaccard}\hlstd{(run1, run2)}
\end{alltt}
\begin{verbatim}
## [1] 0.25
\end{verbatim}
\end{kframe}
\end{knitrout}

The exception to this is Kuncheva's Index.  This requires one additional
parameter, the number of features (e.g. metabolites) in the original dataset.
This metric is designed to account for smaller numbers of variables increasing
the likelihood of matching sets by chance.  Naturally, if you have many more
variables it would be far more indicative of significance if you see the same
small subset again and again as opposed to a small set seeing the same 
variables.

\begin{knitrout}
\definecolor{shadecolor}{rgb}{0.969, 0.969, 0.969}\color{fgcolor}\begin{kframe}
\begin{alltt}
\hlcom{# In this case, 20 original variables}
\hlkwd{kuncheva}\hlstd{(run1, run2,} \hlnum{20}\hlstd{)}
\end{alltt}
\begin{verbatim}
## [1] 0.4
\end{verbatim}
\end{kframe}
\end{knitrout}

\newpage
\subsection{Pairwise Stability}
\subsubsection{Pairwise Feature Stability}

The above examples immediately lead to the question, what if I have more than
two runs?  What I have 3, 5, 10, or more bootstrap iterations?  This is also
know as a data perturbation ensemble approach.  It would be  very tedious to 
have to call the same function for every single comparison.  Therefore a 
pairwise function exists to allow a rapid comparison between all sets.  
This \Rfunction{pairwise.stability} is very similar to the individual 
stability functions in practice.  Let's take an example consisting of 5 runs.

\begin{knitrout}
\definecolor{shadecolor}{rgb}{0.969, 0.969, 0.969}\color{fgcolor}\begin{kframe}
\begin{alltt}
\hlkwd{set.seed}\hlstd{(}\hlnum{21}\hlstd{)}
\hlcom{# matrix of Metabolites identified (e.g. 5 trials)}
\hlstd{features} \hlkwb{<-} \hlkwd{replicate}\hlstd{(}\hlnum{5}\hlstd{,} \hlkwd{sample}\hlstd{(metabs,} \hlnum{10}\hlstd{))}
\end{alltt}
\end{kframe}
\end{knitrout}

Please note that currently only \Rclass{matrix} objects are accepted by
\Rfunction{pairwise.stability}.  To use the function, you simply pass your 
matrix of variables and stability metric (e.g. sorensen).  The only exception
is when applying Kuncheva's Index where the \Robject{nc} parameter again must
be set (which can be ignored otherwise).  This will return all list containing 
the upper triangular matrix of stability values and an overall average.

\begin{knitrout}
\definecolor{shadecolor}{rgb}{0.969, 0.969, 0.969}\color{fgcolor}\begin{kframe}
\begin{alltt}
\hlkwd{pairwise.stability}\hlstd{(features,} \hlstr{"sorensen"}\hlstd{)}
\end{alltt}
\begin{verbatim}
## $comparisons
##            Resample.2 Resample.3 Resample.4 Resample.5
## Resample.1        0.4        0.4        0.5        0.5
## Resample.2        0.0        0.5        0.5        0.5
## Resample.3        0.0        0.0        0.5        0.5
## Resample.4        0.0        0.0        0.0        0.7
## 
## $overall
## [1] 0.5
\end{verbatim}
\end{kframe}
\end{knitrout}

\newpage
\subsubsection{Pairwise Model Stability}

Now, in the spirit of this package, you may have the alternate approach whereby
you have created several bootstrapped data sets and run a different statistical
model on each data set.  You could compare each one manually, but again to 
avoid such tedious work another function is provided for specifically this
purpose.  Let's take a theoretical example where I have bootstrapped 5 different
data sets and applied two models to each dataset (PLSDA and Random Forest).
Please note that currently only \Rclass{list} objects are accepted by
\Rfunction{pairwise.model.stability}.

\textbf{Note} - here I am only randomly sampling but in practice the each model
would have been trained on the same dataset. 

\begin{knitrout}
\definecolor{shadecolor}{rgb}{0.969, 0.969, 0.969}\color{fgcolor}\begin{kframe}
\begin{alltt}
\hlkwd{set.seed}\hlstd{(}\hlnum{999}\hlstd{)}
\hlstd{plsda} \hlkwb{<-}
    \hlkwd{replicate}\hlstd{(}\hlnum{5}\hlstd{,} \hlkwd{paste}\hlstd{(}\hlstr{"Metabolite"}\hlstd{,} \hlkwd{sample}\hlstd{(metabs,} \hlnum{10}\hlstd{),} \hlkwc{sep}\hlstd{=}\hlstr{"_"}\hlstd{))}
\hlstd{rf} \hlkwb{<-}
    \hlkwd{replicate}\hlstd{(}\hlnum{5}\hlstd{,} \hlkwd{paste}\hlstd{(}\hlstr{"Metabolite"}\hlstd{,} \hlkwd{sample}\hlstd{(metabs,} \hlnum{10}\hlstd{),} \hlkwc{sep}\hlstd{=}\hlstr{"_"}\hlstd{))}

\hlstd{features} \hlkwb{<-} \hlkwd{list}\hlstd{(}\hlkwc{plsda}\hlstd{=plsda,} \hlkwc{rf}\hlstd{=rf)}

\hlcom{# nc may be omitted unless using kuncheva}
\hlkwd{pairwise.model.stability}\hlstd{(features,} \hlstr{"kuncheva"}\hlstd{,} \hlkwc{nc}\hlstd{=}\hlnum{20}\hlstd{)}
\end{alltt}
\begin{verbatim}
## $comparisons
##             Resample.1 Resample.2 Resample.3 Resample.4 Resample.5
## plsda.vs.rf        0.3        0.2        0.5        0.3        0.5
## 
## $overall
## [1] 0.36
\end{verbatim}
\end{kframe}
\end{knitrout}

\newpage
\section{Permuation Analysis}

One additional level of analysis often applied to these datasets is Monte
Carlo Permuations.  For example, I would like to check the chance that
my data can distinguish between the groups by chance.  This can be done by
permuting the groups in the dataset and applying the model on each permuation.
This can be accomplished with \Rfunction{perm.class}, which also provides a 
plot of the classification distribution.  Additionally, one may
be interested in another way to evaluate the importance of variables to the 
distinguishing the groups.  Once again, groups can be permuted, the model
refit to the data and the importance of the variables evaluated.  This is
accomplished with \Rfunction{perm.features}.

\begin{knitrout}
\definecolor{shadecolor}{rgb}{0.969, 0.969, 0.969}\color{fgcolor}\begin{kframe}
\begin{alltt}
\hlcom{# permuate class}
\hlkwd{perm.class}\hlstd{(fits, vars, groups,} \hlstr{"rf"}\hlstd{,} \hlkwc{k.folds}\hlstd{=}\hlnum{5}\hlstd{,}
           \hlkwc{metric}\hlstd{=}\hlstr{"Accuracy"}\hlstd{,} \hlkwc{nperm}\hlstd{=}\hlnum{10}\hlstd{)}


\hlcom{# permute variables/features}
\hlkwd{perm.features}\hlstd{(fits, vars, groups,} \hlstr{"rf"}\hlstd{,}
        \hlkwc{sig.level} \hlstd{=} \hlnum{.05}\hlstd{,} \hlkwc{nperm} \hlstd{=} \hlnum{10}\hlstd{)}
\end{alltt}
\end{kframe}
\end{knitrout}

\newpage
\section{Parallel Analysis}

Given the repetitive nature of this analysis there are ample opportunities to
level the power of parallel computing.  These include \Rfunction{fs.stability},
\Rfunction{fs.ensembl.stability}, \Rfunction{perm.class}, and 
\Rfunction{perm.features} simply by specifying the parameter 
\Rcode{allowParallel = TRUE} in the respective function. However, the parallel
backend must be registered in order to work. There are slight differences 
between operating systems so here are two examples.

For Unix OS, you probably will use \CRANpkg{doMC}

\begin{knitrout}
\definecolor{shadecolor}{rgb}{0.969, 0.969, 0.969}\color{fgcolor}\begin{kframe}
\begin{alltt}
\hlkwd{library}\hlstd{(doMC)}

\hlstd{n} \hlkwb{<-} \hlkwd{detectCores}\hlstd{()}
\hlkwd{registerDoMC}\hlstd{(n)}
\end{alltt}
\end{kframe}
\end{knitrout}

For a Windows OS, you likely with use the \CRANpkg{doSNOW}

\begin{knitrout}
\definecolor{shadecolor}{rgb}{0.969, 0.969, 0.969}\color{fgcolor}\begin{kframe}
\begin{alltt}
\hlkwd{library}\hlstd{(parallel)}
\hlkwd{library}\hlstd{(doSNOW)}

\hlcom{# get number of cores}
\hlstd{n} \hlkwb{<-} \hlkwd{detectCores}\hlstd{()}

\hlcom{# make clusters}
\hlstd{cl} \hlkwb{<-} \hlkwd{makeCluster}\hlstd{(n)}

\hlcom{# register backend}
\hlkwd{registerDoSNOW}\hlstd{(cl)}
\end{alltt}
\end{kframe}
\end{knitrout}

\textbf{NOTE} - remember to stop your clusters on Windows when you are finished 
with \Rcode{stopCluster(cl)}.

\newpage
\begin{knitrout}
\definecolor{shadecolor}{rgb}{0.969, 0.969, 0.969}\color{fgcolor}\begin{kframe}
\begin{alltt}
\hlkwd{sessionInfo}\hlstd{()}
\end{alltt}
\begin{verbatim}
## R version 3.2.0 (2015-04-16)
## Platform: x86_64-w64-mingw32/x64 (64-bit)
## Running under: Windows 7 x64 (build 7601) Service Pack 1
## 
## locale:
## [1] LC_COLLATE=English_United States.1252  LC_CTYPE=English_United States.1252   
## [3] LC_MONETARY=English_United States.1252 LC_NUMERIC=C                          
## [5] LC_TIME=English_United States.1252    
## 
## attached base packages:
## [1] parallel  splines   stats     graphics  grDevices utils     datasets  methods  
## [9] base     
## 
## other attached packages:
##  [1] caTools_1.17.1      glmnet_2.0-2        Matrix_1.2-2        pamr_1.55          
##  [5] cluster_2.0.3       gbm_2.1.1           lattice_0.20-33     survival_2.38-3    
##  [9] e1071_1.6-6         randomForest_4.6-10 DiscriMiner_0.1-29  plyr_1.8.3         
## [13] foreach_1.4.2       OmicsMarkeR_1.1.1   knitr_1.10.5       
## 
## loaded via a namespace (and not attached):
##  [1] Rcpp_0.11.6          compiler_3.2.0       nloptr_1.0.4         formatR_1.2         
##  [5] highr_0.5            bitops_1.0-6         class_7.3-13         iterators_1.0.7     
##  [9] tools_3.2.0          lme4_1.1-8           digest_0.6.8         assertive_0.3-0     
## [13] evaluate_0.7         assertive.base_0.0-1 gtable_0.1.2         nlme_3.1-121        
## [17] mgcv_1.8-7           SparseM_1.6          brglm_0.5-9          proto_0.3-10        
## [21] BradleyTerry2_1.0-6  stringr_1.0.0        gtools_3.5.0         stats4_3.2.0        
## [25] nnet_7.3-10          grid_3.2.0           caret_6.0-52         data.table_1.9.4    
## [29] minqa_1.2.4          car_2.0-25           reshape2_1.4.1       ggplot2_1.0.1       
## [33] magrittr_1.5         scales_0.2.5         codetools_0.2-14     MASS_7.3-43         
## [37] pbkrtest_0.4-2       permute_0.8-4        BiocStyle_1.6.0      colorspace_1.2-6    
## [41] quantreg_5.11        stringi_0.5-5        munsell_0.4.2        chron_2.3-47
\end{verbatim}
\end{kframe}
\end{knitrout}

\end{document}
